%!TEX TS-program = xelatex
% vim: set fenc=utf-8

% -*- coding: UTF-8; -*-
%!TEX encoding = UTF-8
\documentclass[twoside]{CUGThesis}

\RequirePackage{setspace}
%设置命令
\newcommand{\makestatement}[2][0]{
	\clearpage
	\thispagestyle{empty}
	\vspace*{1em}
	\begin{center}
\begin{spacing}{1.25}{
		\hei \sanhao
		学士学位论文原创性声明
}
\end{spacing}
	\end{center}


\begin{spacing}{1.1}
 \song \xiaosihao

本人郑重声明:所呈交的学位论文是本人在导师指导下独立进行研究工作所取得的研究成果。
除了文中特别加以标注引用的内容外,本论文不包含任何其他个人或集体已经
发表或撰写的成果作品。本人完全意识到本声明的法律后果由本人承担。\\
\end{spacing}

	\begin{spacing}{1}
	\rightline{
	\begin{tabular}{r}
		\\
		作者签名:\hspace{10em}
		 \hspace{4em}年 \hspace{2em} 月 \hspace{2em} 日
	\end{tabular}
	}
	\end{spacing}

	\vspace*{3em}

	\begin{center}
	\begin{spacing}{1.25}{
		\hei \sanhao
		学位论文使用授权书
	}
	\end{spacing}
	\end{center}

\begin{spacing}{1.1}
\song \xiaosihao

本学位论文作者完全了解学校有关保障、使用学位论文的规定,同意学校保留并向有关学位论文管理部门或机构送交论文的复印件和电子版,
允许论文被查阅和借阅。本人授权省级优秀学士学位论文评选机构将本学位论文的全部或部分内容编入有关数据库进行检索,
可以采用影印、缩印或扫描等复制手段保存和汇编本学位论文。\\
\end{spacing}
本学位论文属于
	\begin{enumerate}
		\item 保密 $\Box$,在\uline{\makebox[3em]}年解密后适用本授权书。
		\item 不保密 $\Box$。
	\end{enumerate}
	(请在以上相应方框内打``$\surd$'')

	\begin{spacing}{1}
		\rightline{
		\begin{tabular}{r}
			\\ \\ 
			作者签名:\hspace{10em}
			 \hspace{4em}年 \hspace{2em} 月 \hspace{2em} 日
			 \\ \\  
			 导师签名:\hspace{10em}
			 \hspace{4em}年 \hspace{2em} 月 \hspace{2em} 日
		\end{tabular}
		}
		\end{spacing}
	\vspace{4em}
	\clearpage
}

\include{title}

\title{基于IPv6的HTTP/2过渡应用的设计与实现} %论文题目
\author{陈立翔} %作者姓名
\date{\today} %日期,默认当日
\school{计算机学院} %院系名称
\classnum{计算机科学与技术} %专业
\stunum {20151001270} %学号
\instructorone{张峰} %指导教师1姓名
\instructoronelevel{副教授}
\instructortwo{陈伟涛} %指导教师2姓名
\instructortwolevel{副教授}

\usepackage{makecell}
\usepackage{multicol}
\usepackage{multirow}
\usepackage{hyperref}

\begin{document}
	\maketitle
	\makestatement
		
	\begin{cnabstract}{IPv6 协议;HTTP/2 协议;HTTP 网关;反向代理;负载均衡;Go语言; }
		今天,网络早已成为人们生活中不可缺少的部分。同时随着我国网民人口的不断增长,
	互联网环境正在变得愈发复杂。如何用有限的服务器资源应对更大的并发连接,如何让更多的设备
	方便地接入网络,如何确保网络服务的安全性、防御潜在的攻击,这都是新时代各个单位正在面临的挑战。
	在这种情况下,原有的一些网络协议已经无法满足要求。目前,IPv4的公网地址资源已经枯竭,早在2011年
	亚洲地区就已经无法分配到IPv4地址;而HTTP/1.x协议经过若干次修订后依然被诟病效率过低和功能不足。\par
		从1996年开始,一系列关于IPv6协议的RFC标准发表出来,旨在取代原有的IPv4协议。IPv6的地址长度为
	128位,能提供远多于IPv4 的地址数量。在IPv4环境下,将不需要IPv4环境下为了解决地址数量不足而衍生出的NAT等技术,每个设备都能得到
	一个公网IP,有利于物联网等技术的发展。但IPv6技术的推进还需要一段时间,在未来一段时间内我们都将
	面临着IPv4和IPv6共存的互联网环境。\par
		2015年发布的HTTP/2协议则旨在取代HTTP/1.x协议。为了提高网络传输的效率,更有效地利用网络资源,
	该协议增加了二进制分帧,多路复用和头部压缩等新的特性。同时目前浏览器中 HTTP/2 
	实现都强制要求数据通过 HTTPS 加密,而不是像 HTTP/1.x 协议那样
	将数据明文发送。这显著增强了数据传输的安全性,有效避免数据被劫持等安全隐患。\par
		本文对于如何让现有的IPv4环境下仅支持HTTP/1.x协议的Web应用能够在IPv4和IPv6双栈环境下工作并
	支持HTTP/2协议,同时提高服务端的并发性能,开展了以下研究工作:
	\begin{enumerate}
		\item 通过对 Go 语言技术的深入研究,分析其协程和并发模型的技术特点。Go 语言是一门在语言层面
		实现了协程的语言,其实现能够避免操作系统调用产生的开销,单机轻松构建百万级协程,CPU资源的利用率极高,
		是近年来服务端开发中解决高并发问题的重要突破点。
		\item 比较了 RFC 标准中 HTTP/1.x 和 HTTP/2 报文的区别,分析了如何将 HTTP/1.x 的报文加工处理
		成 HTTP/2 的报文。同时结合 Go 语言中原生的 net/http 和 tls 库对报文增加 HTTPS 加密,从而
		兼容目前主流浏览器的实现。
		\item 为了将 HTTP/1.x 应用过渡支持 HTTP/2 ,本文采用 Go 语言技术设计了一个 HTTP 网关。该网关
		通过反向代理技术,在无需对原有应用进行任何修改的情况下,将其升级为支持 HTTP/2 协议,并可在 IPv4和
		IPv6双栈环境下工作。
		\item 基于 Weighted Round Robin 算法和 Consistent Hashing 算法对该网关设计了负载均衡策略,
		使得该网关能够支持横向扩容的 Web 应用,应对更复杂的并发情况。
		\item 对该网关进行了压力测试,对比目前市场上提供反向代理功能的服务器如 Apache, Nginx 等,
		得出该网关在一些情况下能够提高系统并发数和减少响应时间,提升用户体验。并通过火焰图分析该网关在
		实际工作环境中的性能瓶颈。
	\end{enumerate}
		\par
		综上所述,本文设计的基于 Go 语言的 HTTP 反向代理网关,能够作为在IPv4和IPv6双栈环境下的 HTTP/2 过渡应用,
	并且具有较好的并发性能。经过测试,具有一定实用价值。
	\end{cnabstract}
	
	\begin{enabstract}{keyword; keyword; keyword; keyword}
		abstract
	\end{enabstract}
	
	\makeToc
	
	%---------------------------------------------开始正文---------------------------------------------
	
	%--------------------------------------------- 第一章 ---------------------------------------------
	\begin{spacing}{2}
		\section{绪论}
	\end{spacing}
	\subsection{研究背景极其意义}
	第一小节,引用实例\cite{引用},按照格式填写mybib.bib
	\subsection{国内外相关工作介绍}
	第二小节
	\subsection{论文主要工作和章节结构}
	etc
	
	
	%------------------------------------------- 第二章 ---------------------------------------------
	\begin{spacing}{2}
		\section{}
	\end{spacing}
	第二章引言
	\subsection{第二章第一小节}
	第二章第一小节
	\subsubsection{第二章第一小节第三级标题}
	插图实例(图\ref{Fig:example})
	\begin{figure}[!t]
		\centering
		\includegraphics[scale=0.5]{Figures/example.jpg}
		\caption{示例图}
		\label{Fig:example}
	\end{figure}
	
	
	
	%--------------------------------------------- 第三章 ---------------------------------------------
	\begin{spacing}{2}
		\section{第三章}
	\end{spacing}
	引言
	\subsection{第三章第一小节}
	开始列表
	\begin{enumerate}
		\item 条目1
		
		说明文字1
		
		\item 条目2
		
		说明文字2
		
		
	\end{enumerate}
	\subsection{第三章第二小节}
	表示例(表\ref{Tab:test})
	\begin{table}[t]
		\centering
		\caption{表示例}
		\label{Tab:test}
		\scalebox{1.0}{
			\begin{tabular}{ccc}
				\toprule  %添加表格头部粗线
				姓名& 编号& 性别s\\
				\midrule  %添加表格中横线
				Steve Jobs& 001& Male\\
				Bill Gates& 002& Female\\
				\hline \hline
			\end{tabular}
		}
		\vspace{10mm}
	\end{table}
	
	
	%---------------------------------------------  致谢  ---------------------------------------------
	\begin{spacing}{2}
		\section*{致谢}
	\end{spacing}
	\phantomsection
	\addcontentsline{toc}{section}{致谢}
	
	致谢
	
	\clearpage
	%---------------------------------------------参考文献---------------------------------------------
	
	\bibliography{Bibs/mybib}
	\phantomsection
	\addcontentsline{toc}{section}{参考文献}
	
\end{document}