%!TEX TS-program = xelatex
% vim: set fenc=utf-8

% -*- coding: UTF-8; -*-
%!TEX encoding = UTF-8
\documentclass[twoside]{CUGThesis}

\RequirePackage{setspace}
%设置命令
\newcommand{\makestatement}[2][0]{
	\clearpage
	\thispagestyle{empty}
	\vspace*{1em}
	\begin{center}
\begin{spacing}{1.25}{
		\hei \sanhao
		学士学位论文原创性声明
}
\end{spacing}
	\end{center}


\begin{spacing}{1.1}
 \song \xiaosihao

本人郑重声明:所呈交的学位论文是本人在导师指导下独立进行研究工作所取得的研究成果。
除了文中特别加以标注引用的内容外,本论文不包含任何其他个人或集体已经
发表或撰写的成果作品。本人完全意识到本声明的法律后果由本人承担。\\
\end{spacing}

	\begin{spacing}{1}
	\rightline{
	\begin{tabular}{r}
		\\
		作者签名:\hspace{10em}
		 \hspace{4em}年 \hspace{2em} 月 \hspace{2em} 日
	\end{tabular}
	}
	\end{spacing}

	\vspace*{3em}

	\begin{center}
	\begin{spacing}{1.25}{
		\hei \sanhao
		学位论文使用授权书
	}
	\end{spacing}
	\end{center}

\begin{spacing}{1.1}
\song \xiaosihao

本学位论文作者完全了解学校有关保障、使用学位论文的规定,同意学校保留并向有关学位论文管理部门或机构送交论文的复印件和电子版,
允许论文被查阅和借阅。本人授权省级优秀学士学位论文评选机构将本学位论文的全部或部分内容编入有关数据库进行检索,
可以采用影印、缩印或扫描等复制手段保存和汇编本学位论文。\\
\end{spacing}
本学位论文属于
	\begin{enumerate}
		\item 保密 $\Box$,在\uline{\makebox[3em]}年解密后适用本授权书。
		\item 不保密 $\Box$。
	\end{enumerate}
	(请在以上相应方框内打``$\surd$'')

	\begin{spacing}{1}
		\rightline{
		\begin{tabular}{r}
			\\ \\ 
			作者签名:\hspace{10em}
			 \hspace{4em}年 \hspace{2em} 月 \hspace{2em} 日
			 \\ \\  
			 导师签名:\hspace{10em}
			 \hspace{4em}年 \hspace{2em} 月 \hspace{2em} 日
		\end{tabular}
		}
		\end{spacing}
	\vspace{4em}
	\clearpage
}

\include{title}

\title{基于IPv6的HTTP/2过渡应用的设计与实现} %论文题目
\author{陈立翔} %作者姓名
\date{\today} %日期,默认当日
\school{计算机学院} %院系名称
\classnum{计算机科学与技术} %专业
\stunum {20151001270} %学号
\instructorone{张峰} %指导教师1姓名
\instructoronelevel{副教授}
\instructortwo{陈伟涛} %指导教师2姓名
\instructortwolevel{副教授}

\usepackage{makecell}
\usepackage{multicol}
\usepackage{multirow}
\usepackage{hyperref}

\begin{document}
	\maketitle
	\makestatement
		
	\begin{cnabstract}{IPv6 协议;HTTP/2 协议;HTTP 网关;反向代理;负载均衡;Go语言; }
		随着互联网的快速发展,网络早已成为人们生活中不可缺少的部分。同时随着我国网民人口的不断增长,
	互联网环境正在变得愈发复杂。如何用有限的服务器资源应对更大的并发连接,如何让更多的设备
	方便地接入网络,如何确保网络服务的安全性、防御潜在的攻击,这都是新时代各个单位正在面临的挑战。
		
	\end{cnabstract}
	
	\begin{enabstract}{keyword; keyword; keyword; keyword}
		abstract
	\end{enabstract}
	
	\makeToc
	
	%---------------------------------------------开始正文---------------------------------------------
	
	%--------------------------------------------- 第一章 ---------------------------------------------
	\begin{spacing}{2}
		\section{绪论}
	\end{spacing}
	\subsection{第一章第一小节}
	第一小节,引用实例\cite{引用},按照格式填写mybib.bib
	\subsection{第一章第二小节}
	第二小节
	\subsection{etc}
	etc
	
	
	%------------------------------------------- 第二章 ---------------------------------------------
	\begin{spacing}{2}
		\section{第二章}
	\end{spacing}
	第二章引言
	\subsection{第二章第一小节}
	第二章第一小节
	\subsubsection{第二章第一小节第三级标题}
	插图实例(图\ref{Fig:example})
	\begin{figure}[!t]
		\centering
		\includegraphics[scale=0.5]{Figures/example.jpg}
		\caption{示例图}
		\label{Fig:example}
	\end{figure}
	
	
	
	%--------------------------------------------- 第三章 ---------------------------------------------
	\begin{spacing}{2}
		\section{第三章}
	\end{spacing}
	引言
	\subsection{第三章第一小节}
	开始列表
	\begin{enumerate}
		\item 条目1
		
		说明文字1
		
		\item 条目2
		
		说明文字2
		
		
	\end{enumerate}
	\subsection{第三章第二小节}
	表示例(表\ref{Tab:test})
	\begin{table}[t]
		\centering
		\caption{表示例}
		\label{Tab:test}
		\scalebox{1.0}{
			\begin{tabular}{ccc}
				\toprule  %添加表格头部粗线
				姓名& 编号& 性别s\\
				\midrule  %添加表格中横线
				Steve Jobs& 001& Male\\
				Bill Gates& 002& Female\\
				\hline \hline
			\end{tabular}
		}
		\vspace{10mm}
	\end{table}
	
	
	%---------------------------------------------  致谢  ---------------------------------------------
	\begin{spacing}{2}
		\section*{致谢}
	\end{spacing}
	\phantomsection
	\addcontentsline{toc}{section}{致谢}
	
	致谢
	
	\clearpage
	%---------------------------------------------参考文献---------------------------------------------
	
	\bibliography{Bibs/mybib}
	\phantomsection
	\addcontentsline{toc}{section}{参考文献}
	
\end{document}